%!TEX program = xelatex
%!BIB program = bibtex

\documentclass[cn,black,12pt,normal]{elegantnote}
\usepackage{float}
\usepackage{hyperref}

\newcommand{\upcite}[1]{\textsuperscript{\textsuperscript{\cite{#1}}}}

\title{课堂作业\\Explore KEGG, IntAct, Reactome}
\author{姜文渊\\1951510}
\institute{School of Life Science, Tongji University}
%\version{1.00}
\date{2021年3月19日}

\begin{document}

\maketitle


\section{问题}
\subsection*{问:}

\textit{Explore KEGG, IntAct, Reactome.}

\textit{Tell the differences between them.}

\subsection*{答:}

\subsection{KEGG}
\textit{KEGG is a database resource for understanding \textbf{high-level functions} and utilities of the biological system...}\upcite{kanehisa2000kegg}

The first impression of KEGG is an old-fashioned user interface with fine designed drawings and manually curated records.

KEGG’s art style changed little during the past ten years, which is like from classical times. 
The database, from my perspective, is meant for human reading rather than for batch process of data-driven tasks.

\subsection{IntAct}
\textit{IntAct provides a freely available, open source database system and analysis tools for \textbf{molecular interaction} data.}\upcite{orchard2014mintact}

IntAct has its style which can be described as STANDARD. 
The visualization of the data is not that straightforward, and most data is presented in a table form.

Data table, as a data form, has its advantage over unstructured data, 
but is not friendly for ordinary users (especially those in the wet lab).

\subsection{Reactome}
\textit{Our goal is to provide \textbf{intuitive} bioinformatics tools for the \textbf{visualization, interpretation and analysis of pathway knowledge} to support basic research, genome analysis, modeling, systems biology and education. }\upcite{fabregat2018reactome}

Interactive features of Reactome brings \textbf{immersive data-explore experience} for users who are not in the field bioinformatics. 

It is knowledge oriented, with coarse-grained fine-grained mixing together, while in the meantime, 
the data are \textbf{hierarchical} organized.

Large scale data mining of this database is also easy, since to download structured data is just few clicks away.

\subsection{Summary}
Difference and features of the three databases are shown in the table below.
Note that most of the comments are subjective opinions of the student.
\begin{table}[H]
    \caption{\textbf{KEGG, IntAct and Reactome}}
    \centering
    \begin{tabular}{cccc}
        \toprule
        Feature&KEGG&IntAct&Reactome\\
        \midrule
        Interactive UI&-&-&+\\
        Structured data table&+&+&-\\
        Manual annotation&+&+&+\\
        Open Source&-&-&+\\
        Convenient data download&-&+&+\\
        \bottomrule
    \end{tabular}
\end{table}

\bibstyle{unsrt}
\bibliography{references}{}
\end{document}
