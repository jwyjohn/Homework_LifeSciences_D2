%!TEX program = xelatex

\documentclass[cn,black,12pt,normal]{elegantnote}
\usepackage{float}
\usepackage{subfigure}




\title{Chia币P盘调优杂谈}
\author{w1118}
\institute{Group of Dmel Research, School of Life Science, Tongji University}
%\version{1.00}
\date{\zhtoday}

\begin{document}

\maketitle

\section{什么是Chia币}
【略,自己去查。】

\section{Chia在\date{\zhtoday}的收益情况}
下面讨论已经p好的盘在\date{\zhtoday}的收益情况。
\begin{itemize}
    \item 日平均收益:1640.403USD/PB
    \item 换算为更亲民的单位:每日10.6 CNY/TB
\end{itemize}

\subsection{乐观估计}
在不考虑全网算力增长和币价波动的情况下,一块2T二手的机械硬盘(成本约180CNY)每天可以带来21元的收益。

\subsection{悲观估计}
考虑到全网算力按指数增长,对一个月的算力情况进行拟合,平均每日算力增长\textbf{1.098}倍。
假定币价不变,利用等比数列求和可知,一块2T二手的机械硬盘共可带来\textbf{110}元的收入,去除成本,纯收益30元。
在这种情况下,约2月后收入就几乎不增长了。此时折价以130元的价格售出这块2T硬盘,
2个月的周期内净收益的悲观估计为60元,等价的年化收益率为203\%。

\section{p盘的几个阶段}
\begin{enumerate}
    \item 计算生成 7 张哈希表,主要使用的算法是前向传播。这一步要处理大量的运算,占用 CPU 最为密集。其实这一步已经生成了足以支持 Proof of Space 的全部数据,只不过效率欠佳,所以还需要后续步骤处理。
    \item 用反向传播算法来清理一遍上面的 7 张哈希表,去除不必要的哈希值,并给表排序。这一步占用 CPU 也较为密集。
    \item 对上一步的结果进行压缩,并将大部分表合并起来。从这一步开始,对 CPU 的消耗降到了较低水平,对内存的占用仍维持在高位,但缓存盘的占用开始逐步下降。
    \item 把剩余的表继续压缩成最终的文件格式,并把文件从缓存盘转移到目标位置。这一步是唯一对目标磁盘进行 IO 操作的。
\end{enumerate}

\section{浅谈调优思路}
调优主要针对两大内容,一是CPU性能,一是IO占用率,通常来说,除了购买新的硬件,两者的上限几乎无法改变。
因而调优的任务在于,使得CPU与IO能够最大程度地配合,减少资源的浪费。
由于各人的硬件设备不能一概而论,所以调优的上限相差较大。
但是,如果合理进行调优,一些常见的硬件可以获得约30\%的提升。

\subsection{获得硬件信息}
Win用户可以通过设备管理器和各类测评软件了解其硬件的型号和性能。
Linux用户可以通过\lstinline{lscpu},\lstinline{lsblk},\lstinline{free}等工具了解其硬件的规格。

需要注意的是以下几个方面:
\begin{itemize}
    \item CPU核心数与线程数
    \item CPU单核性能(下面以CPU-Z跑分为简单判定的指标)
    \item 内存大小
    \item HDD的容量和数量
    \item SSD的容量,读写性能,寿命
\end{itemize}

除了这些硬性参数以外,还要注意设备的散热性能和工作条件,特别是对于机械硬盘,还需避免震动。

\subsection{需要用SSD作为缓存吗}

在\date{\zhtoday},市面上大多数的消费级固态的寿命都不足以完成P盘的任务,
因而除非您的固态是企业级的SLC、MLC或3D Xpoint,不建议为了短期的挖矿行为而损耗日常使用的固态硬盘的寿命。

\textbf{除了SSD作为缓存外,对于拥有阵列卡的用户,利用廉价的机械硬盘组成RAID0作为缓存盘是十分合适的。}

\subsection{关键:确定并行P盘的个数}

对于并行P盘个数的上限,笔者和室友经过测试,\textbf{有以下经验公式(仅适用于k=32的plot)}:
\begin{itemize}
    \item 并行P盘个数的上限=内存GB数/4
    \item 并行P盘个数的上限=线程数
    \item 不使用SSD作为缓存的情况下:并行P盘个数的上限=HDD数*2(每个HDD两个P盘队列)
    \item 使用SSD作为缓存:并行P盘个数的上限=SSD容量/300
\end{itemize}
笔者不推荐使用SSD作为缓存,故而在实际操作中,上面的三条中的最小值为推荐的同时P盘个数。


\subsection{更加精细的操作}

\subsubsection{HDD的内圈与外圈}

一般而言,HDD在进行顺序读写时,内圈速度仅是外圈速度的0.7倍,\textbf{故而对于每块HDD,可划分一个600G的分区,使之在靠外圈的位置(扇区靠前),作为缓存分区。
临时文件储存到缓存分区,最终文件储存到另一个分区,即可使IO效率提高。}

\subsubsection{IO调度的调优}

Win下进行IO调度的调优比较不方便,这里主要讨论Linux的情况。

常见的Linux发行版通常会带有三种IO调度器:
\begin{itemize}
    \item Noop算法
    \item Deadline算法
    \item CFQ算法
\end{itemize}

在新的内核版本中,Deadline算法和CFQ算法都被其对应的多队列的变种所取代(mq-deadline和bfq)。
对于调度器的理论,这里不赘述,仅说明以下测试的结论。

\textbf{HDD使用CFQ或BFQ,对于P盘效率较好。}


\section{两个例子}

\subsection{方案一:桌面ITX大小的p盘机器}
配置要点:
\begin{itemize}
    \item CPU: i5 11400 6C12T 超频至4.2GHz (一定要水冷)
    \item RAM: 16G DDR4 3200MHz
    \item HDD: 6*HGST 4T SATA
    \item P盘策略:一个HDD同时进行两个P盘任务,共12个任务并行
\end{itemize}
性能:一天约P可以生成1.2T的plot文件。

\subsection{方案二:洋垃圾p盘机器}
配置要点:
\begin{itemize}
    \item CPU: QL1M *2 32C64T 1.8GHz
    \item RAM: 128G DDR4 2666MHz
    \item HDD: 16*HGST 4T SATA 使用两块SATA转PCIE扩展卡,配合主板SATA接口
    \item P盘策略:一个HDD同时进行两个P盘任务,共32个任务并行
\end{itemize}
性能:一天约P可以生成3T的plot文件。

\section{特别感谢}

感谢JWY和GYF提供的测试平台与测评数据,愿诸位挖矿顺利。

\end{document}
