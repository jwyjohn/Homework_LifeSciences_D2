\section{Discussion \& Conclusions}

We started our analysis from the multiple sequence alignment of MTF-1 protein sequence in different organisms and find out that protein sequence of MTF-1 is conserved in different organisms. We then analysed the nucleotide sequence of MTF-1, and found that the exon and intron patterns share a high similarity among these organisms.

Similar sequence indicates that similar feathers and functions exist. Therefore we did analysis on the physicochemical properties of MTF-1 in in different organisms, and the results are consistent with our initial guess.

To study the function of MTF-1, we first checked the sequence for special domains. Nuclear localization signal and Zinc finger domain are found, which suggests that MTF-1 works like a transcription factor. Since the zinc finger domains can recognize and bind to specific DNA sequences, we would like to know what the target sequences are. Fortunately, suitable tools are published and we are able to get the consensus sequence of MTF-1 target/binding sequences.

With the MTF-1 target/binding sequences, the question about what genes are regulated via the MTF-1 target/binding sequences (that is, metal response elements MREs) formed. Through sequence screening and transcriptome analysis, several genes that are likely to be regulated by MTF-1 is found.

Still, we do not know the exact structure of the MTF-1, since homologous modeling of MTF-1 fails to give out reasonable structures. \textit{Ab initio} methods, however, cost too much time and computation resources for facilities to afford. Considering that MTF-1 is an unstable protein, elucidation opf its structure by X-ray crystallography may face difficulties, and other methods, like NMR, is not suitable for such a large protein. Cryoelectron Microscopy might be a choice for this task, butin our literature review, no relevant work was found.

If we have the exact structure of the MTF-1, many processes involving MTF-1 will be studied more deeply, and we can gain more insight into the mechanisms of metal regulatory transcription factor, including but not limited to MTF-1.