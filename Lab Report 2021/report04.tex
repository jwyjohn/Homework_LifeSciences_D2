\newpage
\section{Report 04: Stock maintaince}

\subsection*{2021-04-15}

\subsection{Introduction}
		
In this session, we will practise the basic skills commonly used in \textit{Drosophila} experiments. 
	
\subsection{Materials \& Methods} 
		The following protocol is modified based on \textit{Drosophila Maintenance Protocol} \cite{dros2017} according to the preference of our lab.
	\subsubsection{Nutrition} \textit{Drosophila} subsists on the microorganisms, such as yeast, that grow on fermenting fruit. In the lab, the fly’s diet is modified for practical purposes. Several recipes are available, and all contain the essential components: sugar and yeast. The following recipe is adopted in our lab. 
			\begin{enumerate}
				\item Water: \SI{1000}{\milli\liter}
				\item Brown sugar: \SI{135}{\gram}
				\item Agar: \SI{7}{\gram}
				\item Yellow cornmeal: \SI{135}{\gram}
				\item Yeast: \SI{8}{\gram}
				\item Propionic acid: \SI{1000}{\milli\liter}
			\end{enumerate}
			These ingredients are combined, heated, stirred, and then distributed into glass or plastic housing vials. After they are filled, the vials are plugged with a cotton top, wrapped, and cooled.
			
	\subsubsection{Housing} A variety of containers are used for housing \textit{Drosophila}. Vials are used in our lab for fly maintenance and optimally contain 50-100 adults, while other labs may use bottles, which can house 300-600 adult flies, for larger cultures. Incubators control the environment and are capable of holding hundreds of vials and bottles. The normal storage conditions for flies are \SI{25}{\celsius} with  60-65\% relative humidity, but for some stocks (eg. \textit{tub}-${Gal80}^{ts}$ and some special $Gal4$), a temperature of \SI{18}{\celsius} or \SI{29}{\celsius} is needed.
			
			
	\subsubsection{Handling} 
		When working with flies, it is important to practice proper labeling and documentation, and to keep a clean environment to maintain the integrity of fly lines and experiments. A container must be changed when about half of the pupae have eclosed, or left the pupal casing. The casings will appear clear.\par
			
		The pupal stage occurs between the larval and adult stages, and is the time when the larvae incubate and develop into an adult. To identify pupal casings as clear, hold the container up to a light source and inspect the pupa. \par 
			
		Flies are transferred to vials with fresh media via a process known as "flipping flies." Before flipping flies, inspect the media for integrity. Flies cannot survive on food cracked with dryness or contaminated with mold or bacteria. \par 
			
		To flip flies, first, tap the fly vial gently on the counter to knock flies off the sides of the vial. Then quickly remove the stopper, and invert the flies from the old container rapidly into a new one. This process is done rapidly, to prevent flies from escaping or being crushed by the stopper, and to prevent loose flies from entering the vial during flipping. \par 
			
			

	\subsubsection{Anesthetization} Anesthetization is required for sorting flies. Two methods of anesthetization will be discussed here: chilling, and using carbon dioxide.\par
			
		To chill flies place the culture in a \SI{-20}{\celsius} freezer for 8-12 minutes. Then place flies onto a chilled, flat workspace for selection. Flies can also be anesthetized using cold by chilling them directly on a frozen surface. \par 
			
		Carbon dioxide is a preferred method for anesthetization because it does not cause acute mortality in flies. This method is adopted in our lab. The $CO_{2}$ delivery system is made up: of a $CO_{2}$ tank; a tube connected to a needle, to anesthetize flies in vials and bottles; and a tube connected to a $CO_{2}$ plate for analysis under the microscope. \par 
			
		To anesthetize flies, insert the $CO_{2}$ needle through the stopper. Alternatively, tap the container on a surface, remove the stop, and quickly invert the flies onto a $CO_{2}$ plate, keeping a closed seal until the flies are immobile. Using a brush or forceps, gently move the flies into the new container. \par 
			
		To discard unwanted flies, dump them into a fly morgue, which consists of a large bottle filled with isopropanol or ethanol and mineral oil topped with a funnel. \par 
	
	\subsection{Results \& Discussion} 
		In this session, we used the \textit{w1118} stock for practising the skills mentioned above. Most of the operations are easy to perform except the "flipping flies", which takes time to get dexterity. 
