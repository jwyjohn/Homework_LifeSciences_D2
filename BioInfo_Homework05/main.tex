%!TEX program = xelatex
%!BIB program = bibtex

\documentclass[cn,black,12pt,normal]{elegantnote}
\usepackage{float}
\usepackage{hyperref}
\usepackage{amsmath}
\usepackage{amsfonts}
\usepackage{amssymb}


\newcommand{\upcite}[1]{\textsuperscript{\textsuperscript{\cite{#1}}}}

\title{生物信息学作业05}
\author{姜文渊\\1951510}
\institute{School of Life Science, Tongji University}
%\version{1.00}
\date{2021年4月25日}

\begin{document}

\maketitle



\section{请解释BLAST算法快速高效的原因}

Blast\upcite{altschul1990basic}算法采用的是一种\textbf{启发式}算法。

简要来说,BLAST的流程和每一步的时间复杂度如下:
\begin{enumerate}
    \item 将query序列打断成子片段,称之为seed words,$O(m)$
    \item 将seed与预先索引好的序列进行比对,$O(1)$
    \item 选择seed连续打分较高的位置采用动态规划算法进行延伸,$O(mn)\leq O\leq O(n^2)$
    \item 延伸过程也会进行打分,当打分低于某一限度这一延伸过程就会被终止抛弃
    \item 产生了一系列的高得分序列,使用E-value对其显著性进行评估,选出比对结果最好的序列
\end{enumerate}
$n$为数据库大小,$m$为query序列的长度。
在一般情况下,有$n\ggg m$,故平均来看,$O(n)$是BLAST的时间复杂度。
即其查询时间和数据库大小的一次方成正比。

由于BLAST是启发式算法,故很难得到其时间复杂度的精确表达式。
相比于逐个序列进行pairwise比对,BLAST的seed的优化虽说是常数级别的优化,但对于一个长度为7的核苷酸word,
从概率的角度来看,其可以仅进行$1/16384$的pairwise对比。而对于每一组pairwise对比,由于seeding的优化,
其计算量也会在$O(m^2)$的基础上进行一个极大的常数的优化。故而可以达到较快的速度。

此外,BLAST对于数据库里的序列的处理,如预先建立hash表等索引,去除简单重复序列等,也对速度有极大的贡献。

\section{请解释S score, bit score, E value, P value的区别}

\subsection{S score}
S score 的定义如下式所示:

\begin{equation}
S=\sum_{i=1}^{L} s_{r_{1,i},r_{2,i}}
\end{equation}
$s_{i,j}$ 是打分矩阵中的元素。

S score是一种朴素的用于评估序列的生物学相关性的值,其含义与Similarity和Identity类似,
不具有统计学上特殊的含义。

\subsection{Bit score}
Bit score 的定义如下式所示:

\begin{equation}
S^\prime =\frac{\lambda S-ln K}{ln2} 
\end{equation}
$S$是原始分数。 参数$\lambda$和$K$取决于打分矩阵和空位罚分。

在BLAST中,Bit score是以bit表示的归一化分数。Bit score的概念与序列相似性较为类似,Bit score越大,序列相似性越高。
Bit score的统计学的含义在于,当你在其所代表大小量级的数据库里进行搜索时,能够依概率得到所查询序列。

我拿出来N个随机序列,这N条序列序列中能够找出像目前这种相似性的序列。最后再取$log_2$得到的值。因此,Bit score越大,这种相似性出现的概率就会越低,这种概率是随着Bit score的值的增加呈指数变化的。
这里可以看出Bitscore的值是不依赖于比对使用的数据库的大小的。

例如,如果Bit score是30,则为了依概率匹配到所查询序列,数据库的大小大概为$2^{30}\approx10^{9}$条序列。


\subsection{E value}
E value 的定义如下式所示:

\begin{equation}
E=mn2^{-S^\prime }
\end{equation}
$m,n$是常数,与查询序列的大小和数据库大小相关。


E value(expectation value)是一个对Bit score值进行校正之后的值,这个校正的因素主要是在于考虑到的数据库的大小,因为数据库较大的话产生阳性的概率也会很大。
E value的内在含义是:在一次blast分析过程中,query序列与subject序列有N个hit,根据我们的N来确定E value,也就是在一个随机序列中(随机库的大小与数据库相同),
有多少个hit能够被发现(所谓hit的概念就是能够和当前的query和subject的比对结果一致或者好于当前的query和subject的比对结果)。
这里E value衡量的相关参数有两个影响因素需要考虑:数据库的大小以及随机库,由于是随机库所以对于同一个数据库多次blast的结果可能会出现不一样,但是保守的序列的E value值应该是近似的。

\subsection{P value}
P value 的定义如下式所示:

\begin{equation}
Pval_S^{MSP}=Ke^{\lambda S}=2^{-S^\prime}
\end{equation}
从上面的式子可以看出,E value是对P value的一个修正。

P value是事件偶然发生的概率。在BLAST的情况下,与S score相关的P value是偶然获得的分数$x$至少等于$S$的概率,
即$Pval_S=P(x\geq S)$。


\bibstyle{unsrt}
\bibliography{references}{}
\end{document}
