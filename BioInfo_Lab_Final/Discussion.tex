\section{Discussion}

We started our analysis from the literature review of Aplastic Anemia and find out that T cell-mediated autoimmune disorder is involved in this disease. A transcriptomics analysis on CD3+ T-cell is then performed in order to find the difference between the healthy control and the patients. 

Hundreds of differently expressed genes are found, including many commonly see genes like ATPase Na+/K+ transporting subunit alpha 2. To identify which gene we should focus on, only the top 250 DEGs are put in a GO enrichment. The results of the GO enrichment provides us with a deeper insight of the changes in CD3+ T-cells. However, most of the up regulated genes are common in other hematopoietic disease, so we have to focus on those down regulated genes.

Among the down regulated genes, one gene, the platelet factor 4 (PF4), caught our attention, not only because its logFC or p value, but also its relevance with the symptoms of Aplastic Anemia, which is mostly low platelets. 

Investigation of the platelet factor 4 (PF4) is then done using the skills learnt in this course, including but not limited to database research, sequence analysis via psi-blast and MSA, structure visualization and molecular docking.  Some of the properties of platelet factor 4 are confirmed by these studies, which is consistent with the pathophysiology of Aplastic Anemia.

We found that some immunosupressive agent do not have much affinity to the PF4 by molecular docking ,which indicates current therapy methods may not be the reason why the level of PF4 is low.

After studying the features of the platelet factor 4, we realized that lack of PF4 can cause the phenotype of the Aplastic Anemia patients, but the reason behind the down regulation of PF4 expression still remains to be elucidated. Therefore, we searched the literature about the expression regulation of PF4 and compared the promoters and enhancers of PF4 and other genes that both share the same GO function and are down regulated in the transcriptomics analysis. The results are consistent with what previous literature reports, that GATA binding hematopoietic transcription fatctors, like RUNX1, FLI1, GATA-1, and GFI1B, are related to the changes in PF4 expression.\cite{aneja2011mechanism}

Still, more problems remain to be solved, like why the low level of PF4 expression happens in certain cells, but not in all T-cells, why the activity of hematopoietic transcription fatctors are repressed, and how can we rescue the low level of PF4 expression through medicine or other therapeutic methods.