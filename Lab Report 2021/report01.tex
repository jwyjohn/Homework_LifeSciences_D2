\newpage
\section{Report 01: Overview of the project}

\subsection*{2021-03-04}

\subsection{Introduction}
The project of this semester involves the application of our \textit{Drosophila melanogaster} model
on some environment issues. 

Persistent organic pollutants (POPs) are toxic chemicals 
that adversely affect human health and the environment around the world. 
Studies have linked POPs exposures to declines, 
diseases, or abnormalities in a number of wildlife species, including certain kinds of fish, birds, and mammals. 
In people, reproductive, developmental, behavioral, neurologic, endocrine, 
and immunologic adverse health effects have been linked to POPs.\cite{ashraf2017persistent}

Evidence has shown that cancer is linked with POPs exposures\cite{hardell2006utero}, but little is known about
the impact of \textbf{Persistent Organic Pollutants (POPs)} on \textbf{tumor migration}, 
which plays a key role in the occurrence of cancer.

In the first session of this course, 
we discussed how to investigate the impact of Persistent Organic Pollutants (POPs) on tumor migration,
with our \textit{Drosophila melanogaster} model.
We have made a brief design of the experiments of this semester and chosen several POPs to study, 
using \textit{Drosophila melanogaster} with activated $Ras^{v12/Ig}$ gene and $eyeful$ \textit{Drosophila}.
We also planned to investigate which of the pathways are involved in the tumor migration
caused by POPs and we wanted to explore the mechanism behind the migration, if possible.

\subsection{Experiment design}
The main idea of our experiments is to construct some \textit{Drosophila} models for the study of tumor migration 
and then expose our \textit{Drosophila} models to different levels of certain POPs to observe the outcomes.

Traditional methods in genetics and molecular biology will be utilized, including the Fluorescence microscopy,
Quantitave reverse transcription polymerase chain reaction,
Western blot, and etc.
Some analytical techniques which are not common in the field of biology may also be used, 
such as the High-performance liquid chromatography (HPLC) and the High-performance gas chromatography (HPGC).

\subsection{Plan for this semester}
The following experiments is planned to do in this semester:
\begin{itemize}
    \item Selection and construct of \textit{Drosophila} models
    \item \textit{Drosophila} crossing
    \item \textit{Drosophila} stock maintaince
    \item Preparation of the POPs sample
    \item \textit{Drosophila} disection
    \item Fluorescence microscopy
    \item RT-qPCR
    %\item Western blot
    \item HPLC/GC
\end{itemize}

\subsection{Results \& Discussion}

In this section, we are introduced to a new project, which involves applying our knowledge and skills to environment issues. The background of this project and the techniques that will be used is shwon above, and more detailed record of the design and implementation of the experiments will be talked in the following reports.