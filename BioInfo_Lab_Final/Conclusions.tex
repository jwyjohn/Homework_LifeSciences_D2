\section{Conclusions}
In this report, we successfully applied most of what we have learnt in this course to perform a systematical bioinformatics analysis on the disease Aplastic Anemia. The analysis we have done indicates that platelet factor 4 in CD3+ cell is related to Aplastic Anemia, and the reasons behind the regulation of platelet factor 4 expression can be of significance to the treatment of this disease. Moreover, immunosupressive agent Prednisone may not directly influence the function of PF4, which suggests that current immunosupressive threapy might be safe to some extent. Since 1900s, we humans have begun to study this threatening disease, but till now, the molecular mechanism of this disease still remains unclear, and there is no effective treatment to Aplastic Anemia. 

However, due to limited access to literature, first-hand data and hardware resources, some analysis could not be done, such as the analysis of deep sequencing data, metabolomics study, SNP detection of certain genes in patients' T-cells, analysis of beta chain variability of T-cell receptor and etc.

We hope that our analysis can give future researchers guidance and insight into the mechanism behind this disease, and we believe that bioinformatics will play a more important role in the study of all types of diseases.