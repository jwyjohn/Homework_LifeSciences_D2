\section*{Supplementary Materials}

\subsection*{Literature used as data source}
\begin{itemize}
    \item Franzke A, Geffers R, Hunger JK, Pförtner S et al. Identification of novel regulators in T-cell differentiation of aplastic anemia patients. BMC Genomics 2006 Oct 19;7:263. PMID: 17052335
    \item Doi T. [Transcription factors responsible for megakaryocyte-specific gene expression]. Yakugaku Zasshi. 2005 Sep;125(9):685-97. Japanese. doi: 10.1248/yakushi.125.685. PMID: 16141689.
    \item Okada Y, Watanabe M, Nakai T, Kamikawa Y, Shimizu M, Fukuhara Y, Yonekura M, Matsuura E, Hoshika Y, Nagai R, Aird WC, Doi T. RUNX1, but not its familial platelet disorder mutants, synergistically activates PF4 gene expression in combination with ETS family proteins. J Thromb Haemost. 2013 Sep;11(9):1742-50. doi: 10.1111/jth.12355. PMID: 23848403.
\end{itemize}

\subsection*{Computing envrionment}
\begin{lstlisting}
> sessionInfo()
R version 4.0.5 (2021-03-31)
Platform: x86_64-pc-linux-gnu (64-bit)
Running under: Ubuntu 20.04.2 LTS

Matrix products: default
BLAS/LAPACK: /usr/lib/x86_64-linux-gnu/openblas-pthread/libopenblasp-r0.3.8.so

locale:
 [1] LC_CTYPE=en_US.UTF-8       LC_NUMERIC=C              
 [3] LC_TIME=en_US.UTF-8        LC_COLLATE=en_US.UTF-8    
 [5] LC_MONETARY=en_US.UTF-8    LC_MESSAGES=C             
 [7] LC_PAPER=en_US.UTF-8       LC_NAME=C                 
 [9] LC_ADDRESS=C               LC_TELEPHONE=C            
[11] LC_MEASUREMENT=en_US.UTF-8 LC_IDENTIFICATION=C       

attached base packages:
 [1] stats4    grid      parallel  stats     graphics 
 [6] grDevices utils     datasets  methods   base     

other attached packages:
 [1] pathview_1.30.1        clusterProfiler_3.18.1
 [3] topGO_2.42.0           SparseM_1.81          
 [5] GO.db_3.12.1           graph_1.68.0          
 [7] org.Hs.eg.db_3.12.0    AnnotationDbi_1.52.0  
 [9] IRanges_2.24.1         S4Vectors_0.28.1      
[11] DOSE_3.16.0            maptools_1.1-1        
[13] sp_1.4-5               factoextra_1.0.7      
[15] ggplot2_3.3.3          FactoMineR_2.4        
[17] VennDiagram_1.6.20     futile.logger_1.4.3   
[19] pheatmap_1.0.12        umap_0.2.7.0          
[21] limma_3.46.0           GEOquery_2.58.0       
[23] Biobase_2.50.0         BiocGenerics_0.36.1   

loaded via a namespace (and not attached):
  [1] readxl_1.3.1         shadowtext_0.0.8    
  [3] backports_1.2.1      fastmatch_1.1-0     
  [5] plyr_1.8.6           igraph_1.2.6        
  [7] splines_4.0.5        BiocParallel_1.24.1 
  [9] digest_0.6.27        htmltools_0.5.1.1   
 [11] GOSemSim_2.16.1      viridis_0.6.1       
 [13] fansi_0.5.0          magrittr_2.0.1      
 [15] memoise_2.0.0        cluster_2.1.1       
 [17] openxlsx_4.2.3       Biostrings_2.58.0   
 [19] readr_1.4.0          graphlayouts_0.7.1  
 [21] matrixStats_0.59.0   askpass_1.1         
 [23] enrichplot_1.10.2    colorspace_2.0-1    
 [25] blob_1.2.1           ggrepel_0.9.1       
 [27] haven_2.4.1          xfun_0.23           
 [29] dplyr_1.0.6          RCurl_1.98-1.3      
 [31] crayon_1.4.1         jsonlite_1.7.2      
 [33] scatterpie_0.1.6     glue_1.4.2          
 [35] polyclip_1.10-0      gtable_0.3.0        
 [37] zlibbioc_1.36.0      XVector_0.30.0      
 [39] car_3.0-10           Rgraphviz_2.34.0    
 [41] abind_1.4-5          scales_1.1.1        
 [43] futile.options_1.0.1 DBI_1.1.1           
 [45] rstatix_0.7.0        Rcpp_1.0.6          
 [47] viridisLite_0.4.0    reticulate_1.20     
 [49] flashClust_1.01-2    foreign_0.8-81      
 [51] bit_4.0.4            DT_0.18             
 [53] httr_1.4.2           htmlwidgets_1.5.3   
 [55] fgsea_1.16.0         RColorBrewer_1.1-2  
 [57] ellipsis_0.3.2       XML_3.99-0.6        
 [59] pkgconfig_2.0.3      farver_2.1.0        
 [61] utf8_1.2.1           tidyselect_1.1.1    
 [63] labeling_0.4.2       rlang_0.4.11        
 [65] reshape2_1.4.4       munsell_0.5.0       
 [67] cellranger_1.1.0     tools_4.0.5         
 [69] cachem_1.0.5         downloader_0.4      
 [71] cli_2.5.0            generics_0.1.0      
 [73] RSQLite_2.2.7        broom_0.7.6         
 [75] stringr_1.4.0        fastmap_1.1.0       
 [77] bit64_4.0.5          tidygraph_1.2.0     
 [79] zip_2.2.0            purrr_0.3.4         
 [81] KEGGREST_1.30.1      ggraph_2.0.5        
 [83] formatR_1.11         KEGGgraph_1.50.0    
 [85] DO.db_2.9            leaps_3.1           
 [87] xml2_1.3.2           compiler_4.0.5      
 [89] rstudioapi_0.13      curl_4.3.1          
 [91] png_0.1-7            ggsignif_0.6.1      
 [93] tibble_3.1.2         tweenr_1.0.2        
 [95] stringi_1.6.2        RSpectra_0.16-0     
 [97] forcats_0.5.1        lattice_0.20-41     
 [99] Matrix_1.3-2         vctrs_0.3.8         
[101] pillar_1.6.1         lifecycle_1.0.0     
[103] BiocManager_1.30.15  bitops_1.0-7        
[105] data.table_1.14.0    cowplot_1.1.1       
[107] qvalue_2.22.0        R6_2.5.0            
[109] gridExtra_2.3        rio_0.5.26          
[111] lambda.r_1.2.4       MASS_7.3-53.1       
[113] assertthat_0.2.1     openssl_1.4.4       
[115] withr_2.4.2          hms_1.1.0           
[117] tidyr_1.1.3          rvcheck_0.1.8       
[119] carData_3.0-4        ggpubr_0.4.0        
[121] ggforce_0.3.3        scatterplot3d_0.3-41
[123] tinytex_0.32        
\end{lstlisting}

The student's hardware running the analysis is a SuperMicro Chassis with \lstinline{Xeon Gold 5117 @2.00GHz} and \lstinline{128GB DDR4@2133MHz RAM}. \textbf{For those who have an RAM less than 64GB, it may be difficult to reproduce the results of the analysis.}

\subsection*{Code for identifying differently expressed genes}
\begin{lstlisting}
#   Differential expression analysis with limma
library(Biobase)
library(GEOquery)
library(limma)
library(umap)
library(pheatmap)
library(dplyr)
library(clusterProfiler)
library(org.Hs.eg.db)
library("FactoMineR")
library("factoextra")
library("maptools")


# load series and platform data from GEO

gset <- getGEO("GSE3807", GSEMatrix =TRUE, AnnotGPL=TRUE)
if (length(gset) > 1) idx <- grep("GPL96", attr(gset, "names")) else idx <- 1
gset <- gset[[idx]]

# make proper column names to match toptable 
fvarLabels(gset) <- make.names(fvarLabels(gset))

# group membership for all samples
gsms <- "00000011"
sml <- strsplit(gsms, split="")[[1]]

# log2 transformation
ex <- exprs(gset)
qx <- as.numeric(quantile(ex, c(0., 0.25, 0.5, 0.75, 0.99, 1.0), na.rm=T))
LogC <- (qx[5] > 100) ||
  (qx[6]-qx[1] > 50 && qx[2] > 0)
if (LogC) { ex[which(ex <= 0)] <- NaN
exprs(gset) <- log2(ex) }

# assign samples to groups and set up design matrix
gs <- factor(sml)
groups <- make.names(c("patients","Healthy"))
levels(gs) <- groups
gset$group <- gs
design <- model.matrix(~group + 0, gset)
colnames(design) <- levels(gs)

fit <- lmFit(gset, design)  # fit linear model

# set up contrasts of interest and recalculate model coefficients
cts <- c(paste(groups[1],"-",groups[2],sep=""))
cont.matrix <- makeContrasts(contrasts=cts, levels=design)
fit2 <- contrasts.fit(fit, cont.matrix)

# compute statistics and table of top significant genes
fit2 <- eBayes(fit2, 0.01)
tT <- topTable(fit2, adjust="fdr", sort.by="B", number=250)

tT <- subset(tT, select=c("ID","adj.P.Val","P.Value","t","B","logFC","Gene.symbol","Gene.title"))
write.table(tT, file=stdout(), row.names=F, sep="\t")

# Visualize and quality control test results.
# Build histogram of P-values for all genes. Normal test
# assumption is that most genes are not differentially expressed.
tT2 <- topTable(fit2, adjust="fdr", sort.by="B", number=Inf)
hist(tT2$adj.P.Val, col = "grey", border = "white", xlab = "P-adj",
     ylab = "Number of genes", main = "P-adj value distribution")

# summarize test results as "up", "down" or "not expressed"
dT <- decideTests(fit2, adjust.method="fdr", p.value=0.05)

# Venn diagram of results
vennDiagram(dT, circle.col=palette())

# create Q-Q plot for t-statistic
t.good <- which(!is.na(fit2$F)) # filter out bad probes
qqt(fit2$t[t.good], fit2$df.total[t.good], main="Moderated t statistic")

# volcano plot (log P-value vs log fold change)
colnames(fit2) # list contrast names
ct <- 1        # choose contrast of interest
volcanoplot(fit2, coef=ct, main=colnames(fit2)[ct], pch=20,
            highlight=length(which(dT[,ct]!=0)), names=rep('+', nrow(fit2)))

# MD plot (log fold change vs mean log expression)
# highlight statistically significant (p-adj < 0.05) probes
plotMD(fit2, column=ct, status=dT[,ct], legend=F, pch=20, cex=1)
abline(h=0)

################################################################
# General expression data analysis
ex <- exprs(gset)

# box-and-whisker plot
ord <- order(gs)  # order samples by group
palette(c("#1B9E77", "#7570B3", "#E7298A", "#E6AB02", "#D95F02",
          "#66A61E", "#A6761D", "#B32424", "#B324B3", "#666666"))
par(mar=c(7,4,2,1))
title <- paste ("GSE3807", "/", annotation(gset), sep ="")
boxplot(ex[,ord], boxwex=0.6, notch=T, main=title, outline=FALSE, las=2, col=gs[ord])
legend("topleft", groups, fill=palette(), bty="n")

# expression value distribution
par(mar=c(4,4,2,1))
title <- paste ("GSE3807", "/", annotation(gset), " value distribution", sep ="")
plotDensities(ex, group=gs, main=title, legend ="topright")

# UMAP plot (dimensionality reduction)
ex <- na.omit(ex) # eliminate rows with NAs
ex <- ex[!duplicated(ex), ]  # remove duplicates
ump <- umap(t(ex), n_neighbors = 4, random_state = 123)
par(mar=c(3,3,2,6), xpd=TRUE)
plot(ump$layout, main="UMAP plot, nbrs=4", xlab="", ylab="", col=gs, pch=20, cex=1.5)
legend("topright", inset=c(-0.15,0), legend=levels(gs), pch=20,
       col=1:nlevels(gs), title="Group", pt.cex=1.5)
library("maptools")  # point labels without overlaps
pointLabel(ump$layout, labels = rownames(ump$layout), method="SANN", cex=0.6)

# mean-variance trend, helps to see if precision weights are needed
plotSA(fit2, main="Mean variance trend, GSE3807")
\end{lstlisting}

\subsection*{Code for drawing heat map}
\begin{lstlisting}
annotation_col = data.frame(
Type = factor(rep(c("Patient", "Healthy"),c(6,2)))
)
colnames(annotation_col)<-c("Sample")
pheatmap(scale(ex[which(row.names(ex) %in% row.names(tT)),]),border=F,show_rownames=F,show_colnames=F,legend = F,annotation_col = annotation_col)
\end{lstlisting}
\subsection*{Code for principal component analysis}
\begin{lstlisting}
res.pca <- PCA(t(scale(ex[which(row.names(ex) %in% row.names(tT)),])))
fviz_pca_ind(res.pca,col.ind = rep(c('Patient','Healthy'),c(6,2)),repel = T, legend.title = "Groups")
\end{lstlisting}

\subsection*{Code for GO enrichment}
\begin{lstlisting}
output <- bitr(tT[which(tT['logFC']<0),]$Gene.symbol,
fromType = 'SYMBOL',
toType = c('ENTREZID','ENSEMBL','REFSEQ'),
OrgDb = 'org.Hs.eg.db')
ego_cc<-enrichGO(gene = output$ENSEMBL,
OrgDb  = org.Hs.eg.db,
keyType = 'ENSEMBL',
ont  = "CC",
pAdjustMethod = "BH",
pvalueCutoff = 0.01,
qvalueCutoff = 0.05)
dotplot(ego_cc,title="EnrichmentGO")
\end{lstlisting}

\subsection*{Online tools and databases used}
\begin{itemize}
    \item PubMed \url{https://pubmed.ncbi.nlm.nih.gov/}
    \item NCBI BLASTp \url{https://blast.ncbi.nlm.nih.gov/Blast.cgi}
    \item CDD/SPARCLE \url{https://www.ncbi.nlm.nih.gov/Structure/cdd/cddsrv.cgi}
    \item CDART Conserved Domain Architecture Retrieval Tool \url{https://www.ncbi.nlm.nih.gov/Structure/lexington/docs/cdart_about.html}
    \item RCSB PDB \url{https://www.rcsb.org/}
    \item ZINC AC \url{https://zinc.docking.org/}
    \item Swissdock \url{http://www.swissdock.ch/}
    \item neXtprot \url{https://www.nextprot.org/}
\end{itemize}