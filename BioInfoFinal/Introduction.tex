\section{Introduction}

In this paper, we will illustrate the skills we have learned from this course, by applying theories and analytical methods to study a gene named \textbf{MTF-1}.

The paper will first give an overview of MTF-1 and introduce the discovery of MTF-1 gene, which is mostly done in the 1990s by the people working in wet labs. Then, the methods used in this paper will be introduced. Next, results of some analysis using techniques taught in the course will be shown in detail, with steps given in the paper or in supplementary material. Finally, a discussion of the analysis and the results will be presented and we will be talking about issues like the merits and demerits of the methods, what the results indicates and how will these analysis inspire subsequent wet lab works.

Some of the contents of the paper will be based on previous studies, while others will be derived from the data and analysis, which could be inaccurate or misleading. We welcome any kind 
criticism of this paper.

\subsection{Overview of MTF-1}

MTF1 is the metal regulatory transcription factor 1.

In Homo sapiens (human), this gene encodes a transcription factor that induces expression of metallothioneins and other genes involved in metal homeostasis in response to heavy metals such as cadmium, zinc, copper, and silver. The protein is a nucleocytoplasmic shuttling protein that accumulates in the nucleus upon heavy metal exposure and binds to promoters containing a metal-responsive element (MRE).\upcite{ncbi:mtf1human}

\subsection{Discovery of the MTF-1}
In organisms, from the simple prokaryotic cells to complex vertebrate, metals plays an important role since many metal ions are involved in enzymatic reaction. 

The term heavy metal comprises a number of essential and nonessential metals. Among the latter cadmium, mercury and lead are toxic even in trace amounts. Although zinc and copper, which are essential heavy metals, are integral parts of proteins, notably enzymes and transcription factors, an excess of these metals is also toxic. Therefore, elaborate systems to import, sequester, store, transport and expel metals have evolved. Important players in metal homeostasis are the \textbf{metallothioneins}. These small proteins with a high content of cysteines can bind and thereby sequester heavy metals.\upcite{gunther2012taste}

\textbf{Metallothioneins} were discovered in 1957 by Margoshes and Vallee as cadmium-binding proteins in preparations of horse kidney.\upcite{margoshes1957cadmium} Transcription of metallothionein genes is upregulated in response to different stimuli, especially heavy metals. A hallmark of the promoters/enhancers of most metallothionein genes are short DNA sequence motifs termed metal response elements (MREs). The identification of these MREs, that share the core consensus sequence TGCRCNC (R=A or G, N= any nucleotide), suggested the existence of a specific transcription factor that regulates metallothionein expression in response to metals.\upcite{stuart1985identification} In 1988 a MRE-binding protein was identified by electrophoretic mobility shift and methylation interference studies.\upcite{westin1988zinc}\upcite{seguin1988detection} It is bound to its cognate DNA motif in a zinc dependent manner \upcite{westin1988zinc} and was termed MTF-1 (for MRE-binding transcription factor-1, more recently also metal-responsive transcription factor-1 or metal regulatory transcription factor-1). In 1993 the cDNA of mouse MTF-1 was cloned which revealed it as a zinc finger protein.\upcite{radtke1993cloned} Human MTF-1 with a length of 753 amino acids was cloned soon thereafter and found to be highly similar but slightly longer at the C-terminus than mouse MTF-1. \upcite{brugnera1994cloning}\upcite{gunther2012taste}