\section{Introduction}

\subsection{Purpose of the revisit}
We aim to demonstrate the skills that learnt from the course by reviewing and reproducing some bioinformatics analysis related to Acquired Aplastic Anemia. We also did some bioinformatics analysis that are not published previously on this topic. We hope that through our work, we can successfully show that we have finished the course and in the meantime, we can provide new insights into this disease by analysis.

\subsection{Requirements from the course}
After going through the practical exercise of this semester, you have acquired some basic skills.
\begin{enumerate}
    \item Searching the databases with interested genes or proteins to get an overall understanding.
    \item Analyzing the sequence characters of both gene and protein.
    \item Performing MSA to find the family or evolutional relationship of several sequences.
    \item Predicting the protein structure.
    \item Basic transcriptome analysis.
\end{enumerate}
Please use these skills do a systematical bioinformatics analysis on covid-19 or any disease which you are interested in, including but not limited to literature review, related proteins, genes and drugs, structure prediction, MSA, etc.

\subsection{Background information}
Aplastic anemia is a disease in which the body fails to produce blood cells in sufficient numbers. Blood cells are produced in the bone marrow by stem cells that reside there. Aplastic anaemia causes a deficiency of all blood cell types: red blood cells, white blood cells, and platelets.\cite{young2018aplastic}\cite{fauci2015harrison}\cite{porter2011merck}

Aplastic anemia's long history has produced confusing terminology. “Anemia” derives from early ability to measure red blood cells in a hematocrit. Most patients have pancytopenia, with decreased platelets and white blood cells. “Aplastic” refers to the inability marrow to form blood, the end organ effect of diverse pathophysiologic mechanisms. Historically, identification of aplastic anemia was post-mortem, and the biopsy remains fundamental to diagnosis.\cite{longo2018aplastic}\cite{peslak2017diagnosis}

The definitive diagnosis is by bone marrow biopsy; normal bone marrow has 30–70\% blood stem cells, but in aplastic anemia, these cells are mostly gone and replaced by fat.\cite{fauci2015harrison}\cite{porter2011merck}

It is more frequent in people in their teens and twenties but is also common among the elderly. It can be caused by heredity, immune disease, or exposure to chemicals, drugs, or radiation. However, in about one-half of cases, the cause is unknown.\cite{fauci2015harrison}\cite{porter2011merck}

First-line treatment for aplastic anaemia consists of immunosuppressive drugs, typically either anti-lymphocyte globulin or anti-thymocyte globulin, combined with corticosteroids, chemotherapy and ciclosporin. Hematopoietic stem cell transplantation is also used, especially for patients under 30 years of age with a related matched marrow donor.\cite{fauci2015harrison}\cite{porter2011merck}

Anemia may lead to feeling tired, pale skin and a fast heart rate. Low platelets are associated with an increased risk of bleeding, bruising and petechiae. Low white blood cells increase the risk of infections.